\PassOptionsToPackage{unicode=true}{hyperref} % options for packages loaded elsewhere
\PassOptionsToPackage{hyphens}{url}
%
\documentclass[
]{article}
\usepackage{lmodern}
\usepackage{amssymb,amsmath}
\usepackage{ifxetex,ifluatex}
\ifnum 0\ifxetex 1\fi\ifluatex 1\fi=0 % if pdftex
  \usepackage[T1]{fontenc}
  \usepackage[utf8]{inputenc}
  \usepackage{textcomp} % provides euro and other symbols
\else % if luatex or xelatex
  \usepackage{unicode-math}
  \defaultfontfeatures{Scale=MatchLowercase}
  \defaultfontfeatures[\rmfamily]{Ligatures=TeX,Scale=1}
\fi
% use upquote if available, for straight quotes in verbatim environments
\IfFileExists{upquote.sty}{\usepackage{upquote}}{}
\IfFileExists{microtype.sty}{% use microtype if available
  \usepackage[]{microtype}
  \UseMicrotypeSet[protrusion]{basicmath} % disable protrusion for tt fonts
}{}
\makeatletter
\@ifundefined{KOMAClassName}{% if non-KOMA class
  \IfFileExists{parskip.sty}{%
    \usepackage{parskip}
  }{% else
    \setlength{\parindent}{0pt}
    \setlength{\parskip}{6pt plus 2pt minus 1pt}}
}{% if KOMA class
  \KOMAoptions{parskip=half}}
\makeatother
\usepackage{xcolor}
\IfFileExists{xurl.sty}{\usepackage{xurl}}{} % add URL line breaks if available
\IfFileExists{bookmark.sty}{\usepackage{bookmark}}{\usepackage{hyperref}}
\hypersetup{
  pdftitle={Advertising},
  pdfauthor={Gokul Shanth Raveendran},
  pdfborder={0 0 0},
  breaklinks=true}
\urlstyle{same}  % don't use monospace font for urls
\usepackage[margin=1in]{geometry}
\usepackage{color}
\usepackage{fancyvrb}
\newcommand{\VerbBar}{|}
\newcommand{\VERB}{\Verb[commandchars=\\\{\}]}
\DefineVerbatimEnvironment{Highlighting}{Verbatim}{commandchars=\\\{\}}
% Add ',fontsize=\small' for more characters per line
\usepackage{framed}
\definecolor{shadecolor}{RGB}{248,248,248}
\newenvironment{Shaded}{\begin{snugshade}}{\end{snugshade}}
\newcommand{\AlertTok}[1]{\textcolor[rgb]{0.94,0.16,0.16}{#1}}
\newcommand{\AnnotationTok}[1]{\textcolor[rgb]{0.56,0.35,0.01}{\textbf{\textit{#1}}}}
\newcommand{\AttributeTok}[1]{\textcolor[rgb]{0.77,0.63,0.00}{#1}}
\newcommand{\BaseNTok}[1]{\textcolor[rgb]{0.00,0.00,0.81}{#1}}
\newcommand{\BuiltInTok}[1]{#1}
\newcommand{\CharTok}[1]{\textcolor[rgb]{0.31,0.60,0.02}{#1}}
\newcommand{\CommentTok}[1]{\textcolor[rgb]{0.56,0.35,0.01}{\textit{#1}}}
\newcommand{\CommentVarTok}[1]{\textcolor[rgb]{0.56,0.35,0.01}{\textbf{\textit{#1}}}}
\newcommand{\ConstantTok}[1]{\textcolor[rgb]{0.00,0.00,0.00}{#1}}
\newcommand{\ControlFlowTok}[1]{\textcolor[rgb]{0.13,0.29,0.53}{\textbf{#1}}}
\newcommand{\DataTypeTok}[1]{\textcolor[rgb]{0.13,0.29,0.53}{#1}}
\newcommand{\DecValTok}[1]{\textcolor[rgb]{0.00,0.00,0.81}{#1}}
\newcommand{\DocumentationTok}[1]{\textcolor[rgb]{0.56,0.35,0.01}{\textbf{\textit{#1}}}}
\newcommand{\ErrorTok}[1]{\textcolor[rgb]{0.64,0.00,0.00}{\textbf{#1}}}
\newcommand{\ExtensionTok}[1]{#1}
\newcommand{\FloatTok}[1]{\textcolor[rgb]{0.00,0.00,0.81}{#1}}
\newcommand{\FunctionTok}[1]{\textcolor[rgb]{0.00,0.00,0.00}{#1}}
\newcommand{\ImportTok}[1]{#1}
\newcommand{\InformationTok}[1]{\textcolor[rgb]{0.56,0.35,0.01}{\textbf{\textit{#1}}}}
\newcommand{\KeywordTok}[1]{\textcolor[rgb]{0.13,0.29,0.53}{\textbf{#1}}}
\newcommand{\NormalTok}[1]{#1}
\newcommand{\OperatorTok}[1]{\textcolor[rgb]{0.81,0.36,0.00}{\textbf{#1}}}
\newcommand{\OtherTok}[1]{\textcolor[rgb]{0.56,0.35,0.01}{#1}}
\newcommand{\PreprocessorTok}[1]{\textcolor[rgb]{0.56,0.35,0.01}{\textit{#1}}}
\newcommand{\RegionMarkerTok}[1]{#1}
\newcommand{\SpecialCharTok}[1]{\textcolor[rgb]{0.00,0.00,0.00}{#1}}
\newcommand{\SpecialStringTok}[1]{\textcolor[rgb]{0.31,0.60,0.02}{#1}}
\newcommand{\StringTok}[1]{\textcolor[rgb]{0.31,0.60,0.02}{#1}}
\newcommand{\VariableTok}[1]{\textcolor[rgb]{0.00,0.00,0.00}{#1}}
\newcommand{\VerbatimStringTok}[1]{\textcolor[rgb]{0.31,0.60,0.02}{#1}}
\newcommand{\WarningTok}[1]{\textcolor[rgb]{0.56,0.35,0.01}{\textbf{\textit{#1}}}}
\usepackage{graphicx,grffile}
\makeatletter
\def\maxwidth{\ifdim\Gin@nat@width>\linewidth\linewidth\else\Gin@nat@width\fi}
\def\maxheight{\ifdim\Gin@nat@height>\textheight\textheight\else\Gin@nat@height\fi}
\makeatother
% Scale images if necessary, so that they will not overflow the page
% margins by default, and it is still possible to overwrite the defaults
% using explicit options in \includegraphics[width, height, ...]{}
\setkeys{Gin}{width=\maxwidth,height=\maxheight,keepaspectratio}
\setlength{\emergencystretch}{3em}  % prevent overfull lines
\providecommand{\tightlist}{%
  \setlength{\itemsep}{0pt}\setlength{\parskip}{0pt}}
\setcounter{secnumdepth}{-2}
% Redefines (sub)paragraphs to behave more like sections
\ifx\paragraph\undefined\else
  \let\oldparagraph\paragraph
  \renewcommand{\paragraph}[1]{\oldparagraph{#1}\mbox{}}
\fi
\ifx\subparagraph\undefined\else
  \let\oldsubparagraph\subparagraph
  \renewcommand{\subparagraph}[1]{\oldsubparagraph{#1}\mbox{}}
\fi

% set default figure placement to htbp
\makeatletter
\def\fps@figure{htbp}
\makeatother

% https://github.com/rstudio/rmarkdown/issues/337
\let\rmarkdownfootnote\footnote%
\def\footnote{\protect\rmarkdownfootnote}

% https://github.com/rstudio/rmarkdown/pull/252
\usepackage{titling}
\setlength{\droptitle}{-2em}

\pretitle{\vspace{\droptitle}\centering\huge}
\posttitle{\par}

\preauthor{\centering\large\emph}
\postauthor{\par}

\predate{\centering\large\emph}
\postdate{\par}

\title{Advertising}
\author{Gokul Shanth Raveendran}
\date{2/17/2020}

\begin{document}
\maketitle

Data Analysis on Advertising Dataset

Loading the libraries

\begin{Shaded}
\begin{Highlighting}[]
\KeywordTok{rm}\NormalTok{(}\DataTypeTok{list=}\KeywordTok{ls}\NormalTok{())}
\KeywordTok{library}\NormalTok{(rio)}
\KeywordTok{library}\NormalTok{(ggplot2) }
\end{Highlighting}
\end{Shaded}

\begin{verbatim}
## Warning: package 'ggplot2' was built under R version 3.6.2
\end{verbatim}

\begin{Shaded}
\begin{Highlighting}[]
\KeywordTok{library}\NormalTok{(corrplot)}
\end{Highlighting}
\end{Shaded}

\begin{verbatim}
## corrplot 0.84 loaded
\end{verbatim}

\begin{Shaded}
\begin{Highlighting}[]
\KeywordTok{library}\NormalTok{(car)}
\end{Highlighting}
\end{Shaded}

\begin{verbatim}
## Loading required package: carData
\end{verbatim}

Importing the dataset and checking scatter plots for the three columns
in the dataset.

\begin{Shaded}
\begin{Highlighting}[]
\NormalTok{advertising=}\KeywordTok{import}\NormalTok{(}\StringTok{"Advertising.csv"}\NormalTok{)}
\KeywordTok{ggplot}\NormalTok{(}\DataTypeTok{data =}\NormalTok{ advertising, }\KeywordTok{aes}\NormalTok{(}\DataTypeTok{x =}\NormalTok{ print, }\DataTypeTok{y =}\NormalTok{ sales)) }\OperatorTok{+}\StringTok{ }
\StringTok{  }\KeywordTok{geom_point}\NormalTok{(}\DataTypeTok{color=}\StringTok{'black'}\NormalTok{) }\OperatorTok{+}
\StringTok{  }\KeywordTok{geom_smooth}\NormalTok{(}\DataTypeTok{method =} \StringTok{"lm"}\NormalTok{, }\DataTypeTok{se =} \OtherTok{FALSE}\NormalTok{)}
\end{Highlighting}
\end{Shaded}

\includegraphics{Advertising_files/figure-latex/unnamed-chunk-2-1.pdf}

\begin{Shaded}
\begin{Highlighting}[]
\KeywordTok{ggplot}\NormalTok{(}\DataTypeTok{data =}\NormalTok{ advertising, }\KeywordTok{aes}\NormalTok{(}\DataTypeTok{x =}\NormalTok{ tv, }\DataTypeTok{y =}\NormalTok{ sales)) }\OperatorTok{+}\StringTok{ }
\StringTok{  }\KeywordTok{geom_point}\NormalTok{(}\DataTypeTok{color=}\StringTok{'black'}\NormalTok{) }\OperatorTok{+}
\StringTok{  }\KeywordTok{geom_smooth}\NormalTok{(}\DataTypeTok{method =} \StringTok{"lm"}\NormalTok{, }\DataTypeTok{se =} \OtherTok{FALSE}\NormalTok{)}
\end{Highlighting}
\end{Shaded}

\includegraphics{Advertising_files/figure-latex/unnamed-chunk-2-2.pdf}

\begin{Shaded}
\begin{Highlighting}[]
\KeywordTok{ggplot}\NormalTok{(}\DataTypeTok{data =}\NormalTok{ advertising, }\KeywordTok{aes}\NormalTok{(}\DataTypeTok{x =}\NormalTok{ online, }\DataTypeTok{y =}\NormalTok{ sales)) }\OperatorTok{+}\StringTok{ }
\StringTok{  }\KeywordTok{geom_point}\NormalTok{(}\DataTypeTok{color=}\StringTok{'black'}\NormalTok{) }\OperatorTok{+}
\StringTok{  }\KeywordTok{geom_smooth}\NormalTok{(}\DataTypeTok{method =} \StringTok{"lm"}\NormalTok{, }\DataTypeTok{se =} \OtherTok{FALSE}\NormalTok{)}
\end{Highlighting}
\end{Shaded}

\includegraphics{Advertising_files/figure-latex/unnamed-chunk-2-3.pdf}

We can see a linear relationship between tv spending and sales, online
and sales but not so much between print and sales

\begin{Shaded}
\begin{Highlighting}[]
\NormalTok{tv.out =}\StringTok{ }\KeywordTok{lm}\NormalTok{(sales}\OperatorTok{~}\NormalTok{tv, }\DataTypeTok{data =}\NormalTok{ advertising)}
\KeywordTok{summary}\NormalTok{(tv.out)}
\end{Highlighting}
\end{Shaded}

\begin{verbatim}
## 
## Call:
## lm(formula = sales ~ tv, data = advertising)
## 
## Residuals:
##     Min      1Q  Median      3Q     Max 
## -8.3860 -1.9545 -0.1913  2.0671  7.2124 
## 
## Coefficients:
##             Estimate Std. Error t value Pr(>|t|)    
## (Intercept) 7.032594   0.457843   15.36   <2e-16 ***
## tv          0.047537   0.002691   17.67   <2e-16 ***
## ---
## Signif. codes:  0 '***' 0.001 '**' 0.01 '*' 0.05 '.' 0.1 ' ' 1
## 
## Residual standard error: 3.259 on 198 degrees of freedom
## Multiple R-squared:  0.6119, Adjusted R-squared:  0.6099 
## F-statistic: 312.1 on 1 and 198 DF,  p-value: < 2.2e-16
\end{verbatim}

\begin{Shaded}
\begin{Highlighting}[]
\NormalTok{online.out =}\StringTok{ }\KeywordTok{lm}\NormalTok{(sales}\OperatorTok{~}\NormalTok{online, }\DataTypeTok{data =}\NormalTok{ advertising)}
\KeywordTok{summary}\NormalTok{(online.out)}
\end{Highlighting}
\end{Shaded}

\begin{verbatim}
## 
## Call:
## lm(formula = sales ~ online, data = advertising)
## 
## Residuals:
##      Min       1Q   Median       3Q      Max 
## -15.7305  -2.1324   0.7707   2.7775   8.1810 
## 
## Coefficients:
##             Estimate Std. Error t value Pr(>|t|)    
## (Intercept)  9.31164    0.56290  16.542   <2e-16 ***
## online       0.20250    0.02041   9.921   <2e-16 ***
## ---
## Signif. codes:  0 '***' 0.001 '**' 0.01 '*' 0.05 '.' 0.1 ' ' 1
## 
## Residual standard error: 4.275 on 198 degrees of freedom
## Multiple R-squared:  0.332,  Adjusted R-squared:  0.3287 
## F-statistic: 98.42 on 1 and 198 DF,  p-value: < 2.2e-16
\end{verbatim}

\begin{Shaded}
\begin{Highlighting}[]
\NormalTok{print.out =}\StringTok{ }\KeywordTok{lm}\NormalTok{(sales}\OperatorTok{~}\NormalTok{print, }\DataTypeTok{data =}\NormalTok{ advertising)}
\KeywordTok{summary}\NormalTok{(print.out)}
\end{Highlighting}
\end{Shaded}

\begin{verbatim}
## 
## Call:
## lm(formula = sales ~ print, data = advertising)
## 
## Residuals:
##      Min       1Q   Median       3Q      Max 
## -11.2272  -3.3873  -0.8392   3.5059  12.7751 
## 
## Coefficients:
##             Estimate Std. Error t value Pr(>|t|)    
## (Intercept) 12.35141    0.62142   19.88  < 2e-16 ***
## print        0.05469    0.01658    3.30  0.00115 ** 
## ---
## Signif. codes:  0 '***' 0.001 '**' 0.01 '*' 0.05 '.' 0.1 ' ' 1
## 
## Residual standard error: 5.092 on 198 degrees of freedom
## Multiple R-squared:  0.05212,    Adjusted R-squared:  0.04733 
## F-statistic: 10.89 on 1 and 198 DF,  p-value: 0.001148
\end{verbatim}

Since all the models have a significant p value, we can reject the null
hypothesis and we can say that there is a relationship between sales and
the independent variables. We can use the R square to assess the
accuracy of the model. R squared explains variablity of in the data,
higher the R square the more closer are the data to the fitted
regression line.

Since the R squared values are low, we can combine them to see if they
have a better explanation.

\begin{Shaded}
\begin{Highlighting}[]
\NormalTok{print.out =}\StringTok{ }\KeywordTok{lm}\NormalTok{(sales}\OperatorTok{~}\NormalTok{print}\OperatorTok{+}\NormalTok{online}\OperatorTok{+}\NormalTok{tv, }\DataTypeTok{data =}\NormalTok{ advertising)}
\KeywordTok{summary}\NormalTok{(print.out)}
\end{Highlighting}
\end{Shaded}

\begin{verbatim}
## 
## Call:
## lm(formula = sales ~ print + online + tv, data = advertising)
## 
## Residuals:
##     Min      1Q  Median      3Q     Max 
## -8.8277 -0.8908  0.2418  1.1893  2.8292 
## 
## Coefficients:
##              Estimate Std. Error t value Pr(>|t|)    
## (Intercept)  2.938889   0.311908   9.422   <2e-16 ***
## print       -0.001037   0.005871  -0.177     0.86    
## online       0.188530   0.008611  21.893   <2e-16 ***
## tv           0.045765   0.001395  32.809   <2e-16 ***
## ---
## Signif. codes:  0 '***' 0.001 '**' 0.01 '*' 0.05 '.' 0.1 ' ' 1
## 
## Residual standard error: 1.686 on 196 degrees of freedom
## Multiple R-squared:  0.8972, Adjusted R-squared:  0.8956 
## F-statistic: 570.3 on 3 and 196 DF,  p-value: < 2.2e-16
\end{verbatim}

This gives us an improved R square of almost 90\%. However you'll notice
that in a simple linear regression between print and sales was
statistically significant however in multiple it is giving us a negative
beta coefficient suggesting that it causes a decrease in sales? To build
a better model lets take a look at the correlation matrix

\begin{Shaded}
\begin{Highlighting}[]
\KeywordTok{cor}\NormalTok{(advertising)}
\end{Highlighting}
\end{Shaded}

\begin{verbatim}
##                tv     online      print     sales
## tv     1.00000000 0.05480866 0.05664787 0.7822244
## online 0.05480866 1.00000000 0.35410375 0.5762226
## print  0.05664787 0.35410375 1.00000000 0.2282990
## sales  0.78222442 0.57622257 0.22829903 1.0000000
\end{verbatim}

\begin{Shaded}
\begin{Highlighting}[]
\KeywordTok{corrplot}\NormalTok{(}\KeywordTok{cor}\NormalTok{(advertising), }\DataTypeTok{method=}\StringTok{"circle"}\NormalTok{)}
\end{Highlighting}
\end{Shaded}

\includegraphics{Advertising_files/figure-latex/unnamed-chunk-5-1.pdf}

We can see that the correlation between tv and sales, online and sales
are 0.7 and 0.57 but correlation between online and print is higher than
between print and sales. What we can interpret by this is that there has
been a tendency to spend more in print in markets where we spend more on
online advertising. To put things in perspective, even though print
doesn't increase sales it, gets recognition for the impact of online
spending on sales when we do a simple linear regression. Therefore
removing it will not alter the regression significantly in predicting
sales.

\begin{Shaded}
\begin{Highlighting}[]
\NormalTok{onlinetv =}\StringTok{ }\KeywordTok{lm}\NormalTok{(sales}\OperatorTok{~}\NormalTok{online}\OperatorTok{+}\NormalTok{tv, }\DataTypeTok{data =}\NormalTok{ advertising)}
\KeywordTok{summary}\NormalTok{(onlinetv)}
\end{Highlighting}
\end{Shaded}

\begin{verbatim}
## 
## Call:
## lm(formula = sales ~ online + tv, data = advertising)
## 
## Residuals:
##     Min      1Q  Median      3Q     Max 
## -8.7977 -0.8752  0.2422  1.1708  2.8328 
## 
## Coefficients:
##             Estimate Std. Error t value Pr(>|t|)    
## (Intercept)  2.92110    0.29449   9.919   <2e-16 ***
## online       0.18799    0.00804  23.382   <2e-16 ***
## tv           0.04575    0.00139  32.909   <2e-16 ***
## ---
## Signif. codes:  0 '***' 0.001 '**' 0.01 '*' 0.05 '.' 0.1 ' ' 1
## 
## Residual standard error: 1.681 on 197 degrees of freedom
## Multiple R-squared:  0.8972, Adjusted R-squared:  0.8962 
## F-statistic: 859.6 on 2 and 197 DF,  p-value: < 2.2e-16
\end{verbatim}

As we can see from the above summary there is not really that much
difference in the R square. The p values indicate that tv and online are
related to sales but there is no proof that print is associated to sales
in the presence of tv and online. According to the above model the beta
coefficients for online and tv are independent of each other, in the
sense that increase in spending for online will only increase sales only
by the beta coefficient for online. The above can be interpreted as for
every \$1000 increase in advertising, sale through online increase by
187 units and by tv 457 units.

\begin{Shaded}
\begin{Highlighting}[]
\KeywordTok{plot}\NormalTok{(advertising}\OperatorTok{$}\NormalTok{sales,onlinetv}\OperatorTok{$}\NormalTok{fitted.values,}
     \DataTypeTok{xlab=}\StringTok{"Sales"}\NormalTok{, }\DataTypeTok{ylab=}\StringTok{"Fittled values"}\NormalTok{,}
     \DataTypeTok{pch=}\DecValTok{19}\NormalTok{,}\DataTypeTok{main=}\StringTok{"Actuals v. Fitted"}\NormalTok{)}
\KeywordTok{abline}\NormalTok{(}\DecValTok{0}\NormalTok{,}\DecValTok{1}\NormalTok{,}\DataTypeTok{col=}\StringTok{"red"}\NormalTok{,}\DataTypeTok{lwd=}\DecValTok{3}\NormalTok{)}
\end{Highlighting}
\end{Shaded}

\includegraphics{Advertising_files/figure-latex/unnamed-chunk-7-1.pdf}

Since we remove print because we saw that it doesn't significantly
affect the regression model because there was no proof that print is
associated to sales in the presence of tv and online, we can check the
synergy/interaction in markets for both tv and online spending because
if we split a fixed budget for tv and online equally it, will
increase/decrease the overall sales compared to just allocating to one
of the two. We can check for interaction between online and tv and how
both together will have its effect on the regression.

\begin{Shaded}
\begin{Highlighting}[]
\NormalTok{regout =}\StringTok{ }\KeywordTok{lm}\NormalTok{(sales}\OperatorTok{~}\NormalTok{online}\OperatorTok{+}\NormalTok{tv}\OperatorTok{+}\NormalTok{online}\OperatorTok{:}\NormalTok{tv, }\DataTypeTok{data =}\NormalTok{ advertising)}
\KeywordTok{summary}\NormalTok{(regout)}
\end{Highlighting}
\end{Shaded}

\begin{verbatim}
## 
## Call:
## lm(formula = sales ~ online + tv + online:tv, data = advertising)
## 
## Residuals:
##     Min      1Q  Median      3Q     Max 
## -6.3366 -0.4028  0.1831  0.5948  1.5246 
## 
## Coefficients:
##              Estimate Std. Error t value Pr(>|t|)    
## (Intercept) 6.750e+00  2.479e-01  27.233   <2e-16 ***
## online      2.886e-02  8.905e-03   3.241   0.0014 ** 
## tv          1.910e-02  1.504e-03  12.699   <2e-16 ***
## online:tv   1.086e-03  5.242e-05  20.727   <2e-16 ***
## ---
## Signif. codes:  0 '***' 0.001 '**' 0.01 '*' 0.05 '.' 0.1 ' ' 1
## 
## Residual standard error: 0.9435 on 196 degrees of freedom
## Multiple R-squared:  0.9678, Adjusted R-squared:  0.9673 
## F-statistic:  1963 on 3 and 196 DF,  p-value: < 2.2e-16
\end{verbatim}

After adding the interaction term we can see a significant increase in
the R squared value and statistical significance of the interaction
term. The results from the summary suggest that this model is superior
to the other models. It explains that, the interaction term explains
most of the variability in sales. Thus we can say that the model isn't
just additive. We can interpret the interaction term beta coefficent as
increase in effectiveness of the online advertisiment for a one unit
increase in tv advertisement and vice versa.

In order for the model to be usable, it needs to conform to certain
assumptions. Let's check for Linearity, Normality and Equality of
Variance.

\begin{Shaded}
\begin{Highlighting}[]
\CommentTok{#Linearity}
\KeywordTok{plot}\NormalTok{(advertising}\OperatorTok{$}\NormalTok{sales,regout}\OperatorTok{$}\NormalTok{fitted.values,}\DataTypeTok{pch=}\DecValTok{19}\NormalTok{,}\DataTypeTok{main=}\StringTok{"Actual v. Fitted Values"}\NormalTok{)}
\KeywordTok{abline}\NormalTok{(}\DecValTok{0}\NormalTok{,}\DecValTok{1}\NormalTok{,}\DataTypeTok{col=}\StringTok{"red"}\NormalTok{,}\DataTypeTok{lwd=}\DecValTok{3}\NormalTok{)}
\end{Highlighting}
\end{Shaded}

\includegraphics{Advertising_files/figure-latex/unnamed-chunk-9-1.pdf}

\begin{Shaded}
\begin{Highlighting}[]
\KeywordTok{qqnorm}\NormalTok{(regout}\OperatorTok{$}\NormalTok{residuals,}\DataTypeTok{pch=}\DecValTok{19}\NormalTok{,}\DataTypeTok{main=}\StringTok{"Normality Plot"}\NormalTok{)}
\KeywordTok{qqline}\NormalTok{(regout}\OperatorTok{$}\NormalTok{residuals,}\DataTypeTok{col=}\StringTok{"red"}\NormalTok{,}\DataTypeTok{lwd=}\DecValTok{3}\NormalTok{)}
\end{Highlighting}
\end{Shaded}

\includegraphics{Advertising_files/figure-latex/unnamed-chunk-9-2.pdf}

\begin{Shaded}
\begin{Highlighting}[]
\CommentTok{#Equality of Variances}
\KeywordTok{plot}\NormalTok{(regout}\OperatorTok{$}\NormalTok{fitted.values,}\KeywordTok{rstandard}\NormalTok{(regout),}\DataTypeTok{pch=}\DecValTok{19}\NormalTok{,}
     \DataTypeTok{main=}\StringTok{"Standardized Residuals"}\NormalTok{)}
\KeywordTok{abline}\NormalTok{(}\DecValTok{0}\NormalTok{,}\DecValTok{0}\NormalTok{,}\DataTypeTok{col=}\StringTok{"red"}\NormalTok{,}\DataTypeTok{lwd=}\DecValTok{3}\NormalTok{)}
\end{Highlighting}
\end{Shaded}

\includegraphics{Advertising_files/figure-latex/unnamed-chunk-9-3.pdf}

\begin{Shaded}
\begin{Highlighting}[]
\KeywordTok{sqrt}\NormalTok{(}\KeywordTok{mean}\NormalTok{((regout}\OperatorTok{$}\NormalTok{residuals)}\OperatorTok{^}\DecValTok{2}\NormalTok{))}
\end{Highlighting}
\end{Shaded}

\begin{verbatim}
## [1] 0.9340326
\end{verbatim}

From plot 1, fitted vs sales, we can see that it is linear. From plot 2,
we can see that the residuals are almost normal From plot 3, we can see
that it is almost homoscedastic.

\#\#Questions: What statistical method will you use for your analysis
and why?

For my analysis I will use a linear regression. Firstly, the scatter
plots suggest a linear increase between sales and the other independent
variables. After modeling and testing the model for regression
assumptions, we can see that it satisfies linearity, normality and
homoscedasity. Thus I think regression is an appropriate model.

How do you know that your model is good enough? With an R square of 96
and an RMSE of 0.93 I think the model is good enough, because 96\% of
variance is explained. The model accuracy can be explained by the high R
squared value

What do the ``standard errors'' of beta coefficients mean? The standard
error of a coefficient, tells us how much sampling variation there is if
we resample and re estimate the coefficient.

How did the model compute ``degrees of freedom''? Degrees of freedom is
the number of independent pieces of information that went into
calculating the estimate. It is computed by subractng the total number
of observations minus the number of parameters used. In our case we have
3 beta coefficients and the Intercept, therefore the degrees of freedom
is 196.

What is the best measure of model fit in this problem? The best measure
of model fit is R square, Root mean squared deviation and the F test.

What are ``residuals'' in your model and why are they important? The
residuals in the model are the difference between the actual value vs
the value on the regression line. They are important because, residuals
need to follow the assumptions of linearity, normality and equality of
variance to validate our model. Residuals help us see outliers, if the
relationship is linear or non-linear and if we have overlooked groups of
observations.

\end{document}
